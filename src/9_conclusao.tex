\chapter{Conclusão}

Chegando ao fim desta dissertação, conseguimos oferecer uma camada de \textit{middleware} adequada aos requisitos explicitados, capaz de ligar diferentes tipos de dispositivos a aplicações clientes, possuindo mecanismos de automatização avançados, mantendo ao mesmo tempo uma arquitetura escalável e com boa capacidade de manutenção, assegurando a sua extensibilidade para suportar um número maior de dispositivos.

Conseguimos separar dois conceitos muito importantes, a camada de compatibilidade entre os vários dispositivos de múltiplos fabricantes, todos com APIs e protocolos variados, e, a camada de gestão conceptual, que é responsável pelo armazenamento dos dados dos dispositivos e pela automatização e gestão dos mesmos. Isto torna o sistema mais simples de manter, devido à separação destes conceitos, que torna a base de código mais concisa.

Conseguimos cumprir os nossos objetivos, começando pelo estudo das várias soluções IoT existentes, baseando-nos nas mais-valias das mesmas para desenvolver um \textit{middleware} que opera na \textit{cloud}, que tem capacidade para fornecer um \textit{gateway} comum para os dispositivos nas casas dos utilizadores, de fácil extensibilidade devido aos mecanismos de meta-programação utilizados. Por fim, demonstramos todas as funcionalidades utilizando uma aplicação cliente desenvolvida para esse efeito, demonstrando a interoperabilidade do sistema.

\paragraph*{Trabalho Futuro}

Este projeto possui várias áreas candidatas a melhorias, passando desde já pelo software de \textit{tunneling}, o \textit{ngrok}, que possui aquele problema dos URLs aleatórios que já vimos anteriormente, e os sistemas de segurança e de autenticação, quer entre o \textit{hub} e o \textit{middleware}, quer entre as várias aplicações clientes e o \textit{middleware}. 

A partilha de casas entre utilizadores também seria algo a ver no futuro, recorrendo a um sistema de partilha avançado, com permissões e \textit{roles}. A própria autenticação da API do \textit{middleware} poderia ser melhorada, utilizando \textit{OAuth} em vez do sistema de \textit{tokens} simplista que se usa de momento. Utilizando \textit{OAuth} teríamos mais controlo sobre os aspetos de autenticação, melhorando ainda a integração com outros sistemas, dado que este protocolo é bastante utilizado na web.

Além disto, a própria arquitetura do \textit{hub} poderia ser alvo de alterações mais tarde. De momento cada fabricante, ou utilizadores com algum \textit{background} técnico, teriam eles próprios de implementar as classes para os serviços dos dispositivos, essencialmente programando o \textit{hub} para o efeito. Uma alternativa seria o desenvolvimento de uma DSL, que permitisse o desenvolvimento rápido de uma camada de compatibilidade entre o \textit{middleware} e um novo tipo de dispositivo. Também poderia ser desenvolvido outro mecanismo de meta-programação para efetuar \textit{autoload} dos serviços de dispositivos em vez de os especificar manualmente nas \textit{factories}.

Por fim, a própria performance do sistema poderia ser um fator de risco, uma vez que a API do \textit{middleware} foi desenvolvida em \textit{Ruby}, uma linguagem que é conhecida pelos seus ocasionais problemas de performance. Mais tarde poderia-se reescrever esse componente em \textit{Elixir}, uma linguagem baseada em \textit{Erlang}, ideal para ambientes distribuídos e de alta disponibilidade. Em \textit{Elixir} teríamos acesso à framework \textit{Phoenix}, muito semelhante ao \textit{Rails}, podendo efetuar esta conversão de linguagem com menos esforço. Outra alternativa seria apenas reescrever partes vitais do sistema, nomeadamente o processamento de tarefas e o módulo de \textit{websockets}, que são os componentes mais propensos a problemas de performance. Apesar de tudo isto, toda a arquitetura é capaz de escalar horizontalmente.

\paragraph*{Lições e Experiência}

Desta dissertação foram adquiridas várias competências quer ao nível de desenho de sistemas de grande escala, a nível de complexidade, de interoperabilidade serviços, devido à interação \textit{hub}-\textit{middleware}. Também se entrou em contacto com inúmeras tecnologias, \textit{Ruby}, \textit{Java} e \textit{Typescript} são algumas delas, assim como várias técnicas de desenvolvimento, como \textit{model driven development} e meta-programação.

Foi melhorado o conhecimento sobre redes, infraestruturas e disponibilidade de serviço. Também se aprofundou o conhecimento sobre vários protocolos como o HTTP e CoAP, assim como tópicos importantes relativos à IoT, nomeadamente a segurança, talvez um dos mais importantes.

Com esta investigação, conseguimos encontrar um mecanismo com muito potencial, \textit{network tunneling}, neste caso recorrendo ao \textit{ngrok}, uma alternativa segura e fiável a algumas das tecnologias utilizadas, como o UPNP, por exemplo. Apesar do software utilizado ser uma ferramenta para desenvolvimento apenas, a sua alternativa \textit{enterprise} provou ser uma tecnologia muito promissora que encaixava perfeitamente nos requisitos do \textit{middleware}.

\paragraph*{Finalização}

Por fim, conseguimos desenvolver um sistema robusto, simplista, com espaço para melhoria como já vimos, mas que consegue cumprir os objetivos principais com bons fundamentos arquiteturais e de desenvolvimento. Conseguimos oferecer uma API bem organizada e clara, utilizável em qualquer tipo de ambiente devido às escolhas de protocolos adequadas. Esperamos que este trabalho seja útil para o futuro, demonstrando uma abordagem eficaz para o desenvolvimento de soluções IoT que visam à interoperabilidade de dispositivos e disponibilidade dos seus controlos remotamente, de um modo seguro.
