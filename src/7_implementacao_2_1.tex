\subsection{Hub}

Utilizando a framework web \textit{Spark}, conseguimos expor os nossos \textit{ThingServices} via uma API HTTP, como é possível ver no seguinte exemplo:

\begin{minted}{java}
ThingServiceFactory<Light> thingServiceFactory = new LightServiceFactory();
ThingController<Light> controller = new ThingController(
    thingServiceFactory,
    Light.class);

Spark.get("/devices/lights", controller::get);
Spark.put("/devices/lights", controller::put);
\end{minted}

Este exemplo faz uso da tal injeção de dependências explicada durante a fase da arquitetura, conseguindo assim ter apenas um controlador que faz a gestão de todos o tipos de dispositivos. Depois, são criadas duas rotas, uma utilizando o método GET e PUT, que se associam aos respetivos métodos no controlador. Aquela notação faz parte da API do Java 8, que permite passar métodos por referência.

Internamente, os métodos \textit{get} e \textit{put} recebem dois objetos da \textit{framework} utilizada, do tipo \textit{Request} e \textit{Response}, que têm todos os elementos do pedido HTTP proveniente do cliente assim como o objeto que representa a resposta a ser enviada. Com estes objetos é possível manipular cabeçalhos, \textit{query parameters}, entre outros.

No entanto, apesar dos mecanismos arquiteturais utilizados para simplificar o processo de adição de tipos e subtipos de dispositivos, ainda há um problema, sempre que é criado um novo tipo de dispositivo é preciso criar estas rotas manualmente, estando a aumentar a complexidade do código com \textit{boilerplate code}, código repetido sem grande significado. Portanto recorremos a elementos de meta-programação para resolver esta duplicação de código.

Meta-programação consiste então nas muitas maneiras que um programa consegue manipular o seu próprio funcionamento durante o tempo de execução. Em muitas linguagens orientadas a objetos, como o Java por exemplo, existe um mecanismo muito usado de meta-programação, chamado de reflexões, que permite a uma aplicação obter informações sobre as classes e elementos que a compõem, por exemplo, obter uma lista com o nome dos métodos que uma classe define, ou obter todas as classes que implementam um dado interface ou que estendem uma dada classe.

Para resolver este pequeno problema decidiu-se recorrer à biblioteca de \textit{reflections} do Java, que permite inferir durante o \textit{runtime}, a lista de classes que implementa o interface \textit{Thing}. Isto é bastante útil porque podemos utilizar os métodos definidos no interface, \textit{getFactory} e \textit{getStringRepresentation}, para automaticamente criar todas as rotas necessárias para o funcionamento do \textit{hub}.

Primeiro foi criada uma classe auxiliar com um método \textit{static} para obter todas as \textit{Things} do projeto.

\begin{minted}{java}
class ClassDiscoverer {
  public static Set<Class<? extends Thing>> getThings() {
    Reflections reflections = new Reflections("homewatch.things");
    return reflections.getSubTypesOf(Thing.class);
  }
}
\end{minted}

Este método obtêm todas as classes que implementam o interface \textit{Thing} e que estejam presentes no pacote no pacote \textit{homewatch.things}, esta pesquisa por pacote permite reduzir o tempo de execução da mesma.

Depois utilizamos este método para criar as rotas correspondentes aos 5 tipos de dispositivos que se implementaram.

\begin{minted}{java}
private static void deviceControllers() {
  ClassDiscoverer.getThings().forEach(klass -> {
    try {
      Thing t = klass.newInstance();
      ThingServiceFactory thingServiceFactory = t.getFactory();
      ThingController controller = new ThingController(thingServiceFactory, klass);

      Spark.get("/devices/" + t.getStringRepresentation(), controller::get);
      Spark.put("/devices/" + t.getStringRepresentation(), controller::put);
    } catch (InstantiationException | IllegalAccessException e) {
      LoggerUtils.logException(e);
    }
  });
}
\end{minted}

Este método é chamado nas classes de \textit{setup} do projeto, que tratam do processo de inicialização e configuração do pequeno servidor, e essencialmente percorre a lista de classes retornada pelo método \textit{getThings}, utilizando os dois métodos utilitários, \textit{getFactory} e \textit{getStringRepresentation} para criar as rotas de acesso aos serviços dinamicamente. Desta maneira, desde que os interfaces dos dispositivos retornem a \textit{factory} respetiva e afixando a descrição do dispositivo à rota de acesso.

Exemplificando o funcionamento deste mecanismo, se tivermos um dispositivo \textit{Light} onde o método \textit{getFactory} retorna um \textit{LightServiceFactory} e onde o método \textit{getStringRepresentation} retorna \textit{''lights''}, irá ser criado um \textit{controller} utilizando um \textit{LightServiceFactory}, e depois serão criadas duas rotas, \textit{''/devices/lights''} com os métodos GET e PUT.

Assim, conseguimos automaticamente instanciar todos os \textit{controllers} sem ter que repetir pedaços de código idênticos, que iriam acabar por se tornar muito numerosos com o crescimento de tipos de dispositivos.