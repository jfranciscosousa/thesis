\chapter{Conclusão}

Chegando ao fim desta dissertação, conseguimos oferecer uma cada de \textit{middleware} adequada aos requisitos explicitados, capaz de ligar diferentes tipos de dispositivos a aplicações clientes, possuindo mecanismos de automatização avançados, mantendo ao mesmo tempo uma arquitetura escalável e com boa manutenibilidade, assegurando a sua extensibilidade para suportar um número maior de dispositivos.

Conseguimos separar dois conceitos muito importantes, a camada de compatibilidade entre os vários dispositivos de múltiplos fabricantes, todos com APIs e protocolos variados, e, a camada de gestão conceptual, que é responsável pelo armazenamento dos dados dos dispositivos e pela automatização e gestão dos mesmos. Isto torna o sistema mais simples de manter, reduzindo muito a complexidade do mesmo sem passar por grandes esforços.

\paragraph*{Trabalho Futuro}

No entanto, este projeto possui várias áreas candidatas a melhorias, passando desde já pelo software de \textit{tunneling}, o \textit{ngrok}, que tem o tal problema dos URLs aleatórios, os sistemas de segurança e de autenticação, quer entre o \textit{hub} e a \textit{api}, quer entre as várias aplicações clientes e a \textit{api}. 

A partilha de casas entre utilizadores também seria algo a ver no futuro, recorrendo a um sistema de partilha avançado, com permissões e \textit{roles}. A própria autenticação da API do \textit{middleware} poderia ser melhorada, utilizando \textit{OAuth} em vez do sistema de \textit{tokens} simplista que se usa de momento. Utilizando \textit{OAuth} teríamos mais controlo sobre os aspetos de autenticação, melhorando ainda a integração com outros sistemas, dado que este protocolo é bastante utilizado na web.

Além disto, a própria arquitetura do \textit{hub} poderia ser alvo de alterações mais tarde, de momento cada fabricante, ou utilizadores com algum \textit{background} técnico, teriam que eles próprios implementar as classes para os serviços dos dispositivos, essencialmente programando o \textit{hub} para o efeito. Uma alternativa seria o desenvolvimento de uma DSL, que permitisse o desenvolvimento rápido de uma camada de compatibilidade entre o \textit{middleware} e um novo tipo de dispositivo.

Por fim, a própria performance do sistema poderia ser um fator de risco, uma vez que a API do \textit{middleware} foi desenvolvida em \textit{Ruby}, uma linguagem que é conhecida pelos seus ocasionais problemas de performance. Mais tarde poderia-se reescrever esse componente em \textit{Elixir}, uma linguagem baseada em \textit{Erlang}, ideal para ambientes distribuidos e de alta disponibilidade. Em \textit{Elixir} teriamos acesso à framework \textit{Phoenix}, muito semelhante ao \textit{Rails}, podendo efetuar esta conversão de linguagem com menos esforço. Outra alternativa seria apenas reescrever partes vitais do sistema, nomeadamente o processamento de tarefas e o módulo de \textit{websockets}, que são os componentes mais propensos a problemas de performance. Apesar de tudo isto, toda a arquitetura é capaz de escalar horizontalmente.

\paragraph*{Lições e Experiência}

Desta dissertação foram adquiridas várias competências quer ao nível de desenho de sistemas de grande escala, a nível de complexidade, de interoperabilidade serviços, devido à interação \textit{hub}-\textit{api}. Também se entrou em contacto com inúmeras tecnologias, \textit{Ruby}, \textit{Java} e \textit{Typescript} são algumas delas, assim como várias técnicas de desenvolvimento, como \textit{model driven development} e meta-programação.

Foi melhorado o conhecimento sobre redes, infraestruturas e disponibilidade de serviço. Também se aprofundou o conhecimento sobre vários protocolos como o HTTP e CoAP, assim como tópicos importantes relativos à IoT, nomeadamente a segurança, talvez um dos mais importantes.
