\section{Tecnologias}
De seguida serão apresentadas as tecnologias utilizadas para implementar a solução. Cada componente de software, o \textit{hub}, a \textit{api}, as simulações dos dispositivos e e aplicação cliente utilizam linguagens e \textit{frameworks} distintas. Aqui, iremos falar das motivações para a sua escolha.

\paragraph*{Hub}
Para o desenvolvimento deste componente recorreu-se à utilização da linguagem de programação Java, recorrendo à \textit{framework} Spark \footnote{\url{http://sparkjava.com}}.

Java foi uma escolha óbvia devido à experiência na linguagem e ao seu poder no que toca a genéricos, que permite atingir grandes níveis de reutilização de código, algo que será essencial neste caso, dado que vários dispositivos partilham o mesmo modo de comunicação por exemplo.

A \textit{framework} Spark foi escolhida devido à necessidade de expor um serviço HTTP sem ser necessário associar a um modelo arquitetural, por exemplo, algumas \textit{frameworks} são feitas para utilizar MVC. Neste caso, esta \textit{framework}, que até pode ser considerado uma biblioteca devido à sua simplicidade, tem apenas como função expor um servidor HTTP.

Além do Spark, também foram utilizadas mais algumas tecnologias de relevo, nomeadamente clientes HTTP e CoAP, Jersey\footnote{\url{https://jersey.github.io/}} e Californium\footnote{\url{https://www.eclipse.org/californium/}} respetivamente, assim como uma biblioteca de reflexões\footnote{\url{https://github.com/ronmamo/reflections}} para desenvolver mecanismos de meta-programação, e por fim a conhecida biblioteca de serialização de objetos Jackson\footnote{\url{https://github.com/FasterXML/jackson}}.

\paragraph*{Api}
Este componente, sendo o mais complexo de todos, necessita de uma \textit{framework} mais evoluída, portanto decidiu-se utilizar \textit{Ruby on Rails} e portanto a linguagem de programação \textit{Ruby}.

Esta \textit{framework} foi uma escolha óbvia, mais uma vez devido à experiência em trabalhos académicos passados, e também devido à simplicidade de utilização. Utilizando \textit{Ruby on Rails} podemos dedicar mais tempo a funcionalidades em vez de detalhes de implementação. Outra motivação é a existência de um bom suporte de \textit{background tasks}, que como já se falou anteriormente, era um desafio a enfrentar, para se poder implementar as tarefas automatizadas.

Utilizando esta \textit{framework}, tarefas como a camada de \textit{ORM}, filtragem de parâmetros dos pedidos HTTP, serialização de objetos para \textit{JSON} são todas tratadas pelo \textit{Rails}, podendo dar assim mais atenção a outros aspetos mais importantes da aplicação.

No ecossistema do Ruby a utilização de dependências externas é extremamente facilitada com o mecanismo das \textit{rubygems}, um módulo que efetua a gestão das dependências. Utilizando este mecanismo, é possível incluir um ficheiro denominado por \textit{Gemfile}, que lista todas as dependências externas da aplicação. Assim incluímos várias dependências essenciais, como o próprio \textit{Rails} por exemplo, o \textit{Delayed Job} que permite uma declaração de tarefas em segundo plano de modo mais simples, assim como a sua extensão \textit{Delayed Cron Job}, que permite definir tarefas com base numa expressão \textit{cron}, muito útil para definir as nossas \textit{Timed Tasks}.

\paragraph*{Simulação de Dispositivos}

Para as simulações de dispositivos recorreu-se à linguagem Javascript com o \textit{runtime serverside} Node.JS, utilizando a framework \textit{express} \footnote{\url{https://expressjs.com/}}. A combinação de \textit{javascript} com o \textit{express} permite muito rapidamente desenvolver servidores web, uma vez que estes simuladores irão fornecer APIs HTTP.

Também se utilizou a biblioteca \textit{coap-router} \footnote{\url{https://github.com/MagicCube/coap-router}}, que possui um funcionamento muito semelhante ao \textit{express}, para criar os dispositivos que utilizam CoAP em vez de HTTP, como protocolo de comunicação.

De seguida, parte do código que compõe a simulação de um dispositivo do tipo \textit{light}
\begin{minted}{js}
  app.get("/status", function (req, res) {
    console.log(`RECEIVED GET REQUEST: ${JSON.stringify(req.body)}`);
    res.send({ power: power });
  });

  app.put("/status", (req, res) => {
    console.log(`RECEIVED PUT REQUEST: ${JSON.stringify(req.body)}`);
    power = req.body.power;
    res.send({ power: power });
  });
\end{minted}

Este pequeno script expõe um serviço numa porta determinada na invocação do mesmo (ou 80 por defeito), que possui duas rotas para o URL \textit{''/status''} uma utilizando o método GET e outra o PUT, permitindo ''desligar'' e ''ligar'' esta luz imaginária manipulando o estado do equipamento através de um objeto JSON que o descreve, como por exemplo:

\begin{minted}{js}
  {
    "power": true
  }
\end{minted}

Foram feitos outros scripts muito semelhantes a este, mudando apenas o formato do objeto, para fechaduras, estações meteorológicas, termostatos e sensores de movimento. Quando a dispositivos que utilizam CoAP, apenas se simulou dispositivos do tipo lâmpada.

\paragraph*{\textit{Wrapper} da \textit{api}}

Este componente, como já foi explicado na arquitetura, providencia um adaptador para a \textit{api} do \textit{middleware} na linguagem da aplicação cliente, que neste caso, vai ser \textit{Javascript}. 

Este componente é bastante simples, tendo apenas que fornecer uma API utilizando a linguagem, onde internamente faz uso da API em HTTP desenvolvida para este \textit{middleware}.

\begin{minted}{js}
class Users {
  constructor(axios) {
    this.axios = axios;
  }

  /**
    * Registers a user
    * @param {Object} user
    * @param {string} user.name
    * @param {string} user.email
    * @param {string} user.password
    * @return {Promise}
    */
  register(user) {
    return this.axios.post("/users", { user });
  };
}
\end{minted}

No \textit{wrapper}, utilizamos a biblioteca \textit{axios}\footnote{\url{https://github.com/axios/axios}}, um cliente HTTP muito utilizado em aplicações desenvolvidas em \textit{Javascript}, para interagir com os \textit{endpoints} da \textit{api} do \textit{middleware}, que opera utilizando o protocolo HTTP.

O \textit{wrapper} oferece uma classe base que contêm vários módulos, um para cada \textit{endpoint} da API. Cada módulo é também uma classe, que é instanciada utilizando o cliente HTTP \textit{axios}.

\begin{minted}{js}
class HomewatchApi {
  constructor(url, cache) {
    this.axios = axios.create({
      baseURL: url,
    });

    if (cache === true) {
      this.axios.get = getFromCache(this.axios.get);
      this.axios.interceptors.response.use(cacheClear);
    }
  }

  set auth(auth) {
    Object.assign(this.axios.defaults, { 
        headers: { authorization: `Bearer ${auth}` } 
    });
  }
  
  /**
   * @return {Users}
   */
  get users() {
    return new Users(this.axios);
  }
}
\end{minted}

A classe base da API pode ser instanciada com dois argumentos, \textit{url} que é uma string que indica o \textit{endpoint} da API, que apenas é variável neste contexto de desenvolvimento. Num ambiente de produção seria o endereço fixo da API, e um valor booleano \textit{cache} que ativa o \textit{caching} dos pedidos GET feitos à API. Também temos um atributo \textit{auth}, que deve conter um \textit{token} de autenticação devolvido pelo método \textit{login} da API.

De seguida poderemos então ver em ação o funcionamento deste \textit{wrapper}:

\begin{minted}{js}
  async function exemplo(){
    const jose = {
      name: "jose",
      email: "jose@mail.com",
      password: "123456",
    };
    
    // login
    let user = await homewatch.users.login(jose);
    homewatch.auth = user.data.jwt;

    // current user
    let currentUser = await homewatch.users.currentUser();

    // list user's homes
    let homes = await homewatch.homes.listHomes();
  }
\end{minted}

Todos os métodos do \textit{wrapper} devolvem \textit{Promises}, um mecanismo que facilita a programação assíncrona muito predominante nesta linguagem, devido a esta ser maioritariamente utilizada em contextos de aplicações clientes interativas. Uma \textit{Promise} é um objeto que representa uma operação assíncrona que irá ser completada no futuro, podendo também falhar. Neste exemplo, utilizamos o mecanismo \textit{async/await} introduzido pelo ES6, uma revisão da especificação do \textit{Ecmascript} onde o \textit{Javascript} é uma implementação deste, que facilita imenso a programação assíncrona neste ambiente.

\paragraph*{Aplicação Cliente}

Por fim, foi desenvolvida uma aplicação cliente para efetuar uma demonstração da API utilizando a \textit{framework} \textit{Ionic 2}\footnote{\url{http://ionicframework.com/}}, que recorre a outra \textit{framework} de \textit{frontend} bastante popular chamada de \textit{Angular 2}\footnote{\url{https://angular.io/}}.

O \textit{Ionic} é essencialmente um ajudante à conversão de uma aplicação web para uma aplicação \textit{mobile}, recorrendo ao \textit{Angular} para providenciar um bom ambiente de desenvolvimento de aplicações gráficas ao mesmo tempo que oferece vários mecanismos para interagir com as APIs nativas dos aparelhos \textit{mobile}. Essencialmente o \textit{Ionic} faz uso dos \textit{browsers} embebidos presentes nos 3 sistemas operativos suportados por esta \textit{framework}, \textit{Android}, \textit{iOS} e \textit{Windows Phone}, para apresentar uma aplicação web desenvolvida em \textit{Angular} empacotada numa aplicação ''nativa''.

Como esta \textit{framework} utiliza o \textit{Angular}, temos que recorrer a \textit{Typescript}, uma alternativa ao \textit{Javascript} que basicamente adiciona \textit{type safety} à linguagem. De notar que, apesar de não ser \textit{Javascript}, que é a linguagem onde se desenvolveu o \textit{wrapper}, podemos utilizar-lo na mesma, uma vez que em \textit{Typescript} podemos ter dependências em módulos \textit{Javascript} na mesma.

Esta abordagem oferece algumas vantagens e também desvantagens onde neste caso, as primeiras superam as últimas. É possível desde já desenvolver aplicações para 3 sistemas operativos recorrendo à mesma \textit{codebase}, o que é incrível por si só, permitindo ainda desenvolver as aplicações com linguagens feitas para aplicações gráficas, que é o caso do HTML, JS (Typescript neste caso) e CSS, mais fáceis de utilizar do que as alternativas, Java, Swift e C\#. No entanto, também temos algumas desvantagens, sendo a performance um deles, porque é muito difícil combater com código nativo, uma vez que estamos essencialmente a correr a nossa aplicação num \textit{stripped down browser}, e o acesso às APIs nativas também é um problema, porque apesar de o \textit{Ionic} fornecer \textit{wrappers} a estas, nem todas são acessíveis através dum ambiente como este.

Esta \textit{framework} facilitou imenso o processo de desenvolvimento desta aplicação, que não era o foco da dissertação, mas permitiu mesmo assim apresentar algo onde se pudesse testar toda a aplicação num contexto mais semelhante ao mundo real, onde um utilizador pudesse realmente usufruir das abordagens aqui feitas. Neste capítulo não vamos abordar mais nem o \textit{wrapper} nem esta aplicação, deixando isso para o capítulo 8, onde vamos ver a aplicação em ação de modo a que as funcionalidades do \textit{middleware} fiquem mais claras.