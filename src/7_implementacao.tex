\chapter{Implementação}

Neste capítulo irá ser explicado todo o processo da implementação, desde as escolhas das tecnologias e às metodologias de desenvolvimento aos detalhes e pormenores de implementação de todos os componentes.

Após a definição da arquitetura, foram criados no sistema de controlo de versões remoto \textit{Github} para a implementação da plataforma:
\begin{itemize}
    \item Simuladores de dispositivos - \url{https://github.com/um-homewatch/api}
    \item \textit{Hub} - \url{https://github.com/um-homewatch/hub}
    \item \textit{Api} - \url{https://github.com/um-homewatch/stub-devices}
    \item \textit{Wrapper} da \textit{api} - \url{https://github.com/um-homewatch/js-wrapper}
    \item Aplicação cliente - \url{https://github.com/um-homewatch/app}
\end{itemize}

A ordem dos repositórios também ditam a ordem de implementação dos mesmos, seguindo a metodologia incremental e iterativa, onde os componentes foram implementados incrementalmente. O \textit{hub}, a \textit{api} e o \textit{wrapper} exibem todo o método de \textit{continuous integration}, recorrendo ao \textit{GitHub} e ao \textit{Travis}, enquanto que a \textit{api} é \textit{deployed} no \textit{Heroku}.

\section{Tecnologias}
De seguida serão apresentadas as tecnologias utilizadas para implementar a solução. Cada componente de software, o \textit{hub}, a \textit{api}, as simulações dos dispositivos e e aplicação cliente utilizam linguagens e \textit{frameworks} distintas. Aqui, iremos falar das motivações para a sua escolha.

\paragraph*{Hub}
Para o desenvolvimento deste componente recorreu-se à utilização da linguagem de programação Java, recorrendo à \textit{framework} Spark \footnote{\url{http://sparkjava.com}}.

Java foi uma escolha óbvia devido à experiência na linguagem e ao seu poder no que toca a genéricos, que permite atingir grandes níveis de reutilização de código, algo que será essencial neste caso, dado que vários dispositivos partilham o mesmo modo de comunicação por exemplo.

A \textit{framework} Spark foi escolhida devido à necessidade de expor um serviço HTTP sem ser necessário associar a um modelo arquitetural, por exemplo, algumas \textit{frameworks} são feitas para utilizar MVC. Neste caso, esta \textit{framework}, que até pode ser considerado uma biblioteca devido à sua simplicidade, tem apenas como função expor um servidor HTTP.

Além do Spark, também foram utilizadas mais algumas tecnologias de relevo, nomeadamente clientes HTTP e CoAP, Jersey\footnote{\url{https://jersey.github.io/}} e Californium\footnote{\url{https://www.eclipse.org/californium/}} respetivamente, assim como uma biblioteca de reflexões\footnote{\url{https://github.com/ronmamo/reflections}} para desenvolver mecanismos de meta-programação, e por fim a conhecida biblioteca de serialização de objetos Jackson\footnote{\url{https://github.com/FasterXML/jackson}}.

\paragraph*{Api}
Este componente, sendo o mais complexo de todos, necessita de uma \textit{framework} mais evoluída, portanto decidiu-se utilizar \textit{Ruby on Rails} e portanto a linguagem de programação \textit{Ruby}.

Esta \textit{framework} foi uma escolha óbvia, mais uma vez devido à experiência em trabalhos académicos passados, e também devido à simplicidade de utilização. Utilizando \textit{Ruby on Rails} podemos dedicar mais tempo a funcionalidades em vez de detalhes de implementação. Outra motivação é a existência de um bom suporte de \textit{background tasks}, que como já se falou anteriormente, era um desafio a enfrentar, para se poder implementar as tarefas automatizadas.

Utilizando esta \textit{framework}, tarefas como a camada de \textit{ORM}, filtragem de parâmetros dos pedidos HTTP, serialização de objetos para \textit{JSON} são todas tratadas pelo \textit{Rails}, podendo dar assim mais atenção a outros aspetos mais importantes da aplicação.

No ecossistema do Ruby a utilização de dependências externas é extremamente facilitada com o mecanismo das \textit{rubygems}, um módulo que efetua a gestão das dependências. Utilizando este mecanismo, é possível incluir um ficheiro denominado por \textit{Gemfile}, que lista todas as dependências externas da aplicação. Assim incluímos várias dependências essenciais, como o próprio \textit{Rails} por exemplo, o \textit{Delayed Job} que permite uma declaração de tarefas em segundo plano de modo mais simples, assim como a sua extensão \textit{Delayed Cron Job}, que permite definir tarefas com base numa expressão \textit{cron}, muito útil para definir as nossas \textit{Timed Tasks}.

\paragraph*{Simulação de Dispositivos}

Para as simulações de dispositivos recorreu-se à linguagem Javascript com o \textit{runtime serverside} Node.JS, utilizando a framework \textit{express} \footnote{\url{https://expressjs.com/}}. A combinação de \textit{javascript} com o \textit{express} permite muito rapidamente desenvolver servidores web, uma vez que estes simuladores irão fornecer APIs HTTP.

Também se utilizou a biblioteca \textit{coap-router} \footnote{\url{https://github.com/MagicCube/coap-router}}, que possui um funcionamento muito semelhante ao \textit{express}, para criar os dispositivos que utilizam CoAP em vez de HTTP, como protocolo de comunicação.

De seguida, parte do código que compõe a simulação de um dispositivo do tipo \textit{light}
\begin{minted}{js}
  app.get("/status", function (req, res) {
    console.log(`RECEIVED GET REQUEST: ${JSON.stringify(req.body)}`);
    res.send({ power: power });
  });

  app.put("/status", (req, res) => {
    console.log(`RECEIVED PUT REQUEST: ${JSON.stringify(req.body)}`);
    power = req.body.power;
    res.send({ power: power });
  });
\end{minted}

Este pequeno script expõe um serviço numa porta determinada na invocação do mesmo (ou 80 por defeito), que possui duas rotas para o URL \textit{''/status''} uma utilizando o método GET e outra o PUT, permitindo ''desligar'' e ''ligar'' esta luz imaginária manipulando o estado do equipamento através de um objeto JSON que o descreve, como por exemplo:

\begin{minted}{js}
  {
    "power": true
  }
\end{minted}

Foram feitos outros scripts muito semelhantes a este, mudando apenas o formato do objeto, para fechaduras, estações meteorológicas, termostatos e sensores de movimento. Quando a dispositivos que utilizam CoAP, apenas se simulou dispositivos do tipo lâmpada.

\paragraph*{\textit{Wrapper} da \textit{api}}

Este componente, como já foi explicado na arquitetura, providencia um adaptador para a \textit{api} do \textit{middleware} na linguagem da aplicação cliente, que neste caso, vai ser \textit{Javascript}. 

Este componente é bastante simples, tendo apenas que fornecer uma API utilizando a linguagem, onde internamente faz uso da API em HTTP desenvolvida para este \textit{middleware}.

\begin{minted}{js}
class Users {
  constructor(axios) {
    this.axios = axios;
  }

  /**
    * Registers a user
    * @param {Object} user
    * @param {string} user.name
    * @param {string} user.email
    * @param {string} user.password
    * @return {Promise}
    */
  register(user) {
    return this.axios.post("/users", { user });
  };
}
\end{minted}

No \textit{wrapper}, utilizamos a biblioteca \textit{axios}\footnote{\url{https://github.com/axios/axios}}, um cliente HTTP muito utilizado em aplicações desenvolvidas em \textit{Javascript}, para interagir com os \textit{endpoints} da \textit{api} do \textit{middleware}, que opera utilizando o protocolo HTTP.

O \textit{wrapper} oferece uma classe base que contêm vários módulos, um para cada \textit{endpoint} da API. Cada módulo é também uma classe, que é instanciada utilizando o cliente HTTP \textit{axios}.

\begin{minted}{js}
class HomewatchApi {
  constructor(url, cache) {
    this.axios = axios.create({
      baseURL: url,
    });

    if (cache === true) {
      this.axios.get = getFromCache(this.axios.get);
      this.axios.interceptors.response.use(cacheClear);
    }
  }

  set auth(auth) {
    Object.assign(this.axios.defaults, { 
        headers: { authorization: `Bearer ${auth}` } 
    });
  }
  
  /**
   * @return {Users}
   */
  get users() {
    return new Users(this.axios);
  }
}
\end{minted}

A classe base da API pode ser instanciada com dois argumentos, \textit{url} que é uma string que indica o \textit{endpoint} da API, que apenas é variável neste contexto de desenvolvimento. Num ambiente de produção seria o endereço fixo da API, e um valor booleano \textit{cache} que ativa o \textit{caching} dos pedidos GET feitos à API. Também temos um atributo \textit{auth}, que deve conter um \textit{token} de autenticação devolvido pelo método \textit{login} da API.

De seguida poderemos então ver em ação o funcionamento deste \textit{wrapper}:

\begin{minted}{js}
  async function exemplo(){
    const jose = {
      name: "jose",
      email: "jose@mail.com",
      password: "123456",
    };
    
    // login
    let user = await homewatch.users.login(jose);
    homewatch.auth = user.data.jwt;

    // current user
    let currentUser = await homewatch.users.currentUser();

    // list user's homes
    let homes = await homewatch.homes.listHomes();
  }
\end{minted}

Todos os métodos do \textit{wrapper} devolvem \textit{Promises}, um mecanismo que facilita a programação assíncrona muito predominante nesta linguagem, devido a esta ser maioritariamente utilizada em contextos de aplicações clientes interativas. Uma \textit{Promise} é um objeto que representa uma operação assíncrona que irá ser completada no futuro, podendo também falhar. Neste exemplo, utilizamos o mecanismo \textit{async/await} introduzido pelo ES6, uma revisão da especificação do \textit{Ecmascript} onde o \textit{Javascript} é uma implementação deste, que facilita imenso a programação assíncrona neste ambiente.

\paragraph*{Aplicação Cliente}

Por fim, foi desenvolvida uma aplicação cliente para efetuar uma demonstração da API utilizando a \textit{framework} \textit{Ionic 2}\footnote{\url{http://ionicframework.com/}}, que recorre a outra \textit{framework} de \textit{frontend} bastante popular chamada de \textit{Angular 2}\footnote{\url{https://angular.io/}}.

O \textit{Ionic} é essencialmente um ajudante à conversão de uma aplicação web para uma aplicação \textit{mobile}, recorrendo ao \textit{Angular} para providenciar um bom ambiente de desenvolvimento de aplicações gráficas ao mesmo tempo que oferece vários mecanismos para interagir com as APIs nativas dos aparelhos \textit{mobile}. Essencialmente o \textit{Ionic} faz uso dos \textit{browsers} embebidos presentes nos 3 sistemas operativos suportados por esta \textit{framework}, \textit{Android}, \textit{iOS} e \textit{Windows Phone}, para apresentar uma aplicação web desenvolvida em \textit{Angular} empacotada numa aplicação ''nativa''.

Como esta \textit{framework} utiliza o \textit{Angular}, temos que recorrer a \textit{Typescript}, uma alternativa ao \textit{Javascript} que basicamente adiciona \textit{type safety} à linguagem. De notar que, apesar de não ser \textit{Javascript}, que é a linguagem onde se desenvolveu o \textit{wrapper}, podemos utilizar-lo na mesma, uma vez que em \textit{Typescript} podemos ter dependências em módulos \textit{Javascript} na mesma.

Esta abordagem oferece algumas vantagens e também desvantagens onde neste caso, as primeiras superam as últimas. É possível desde já desenvolver aplicações para 3 sistemas operativos recorrendo à mesma \textit{codebase}, o que é incrível por si só, permitindo ainda desenvolver as aplicações com linguagens feitas para aplicações gráficas, que é o caso do HTML, JS (Typescript neste caso) e CSS, mais fáceis de utilizar do que as alternativas, Java, Swift e C\#. No entanto, também temos algumas desvantagens, sendo a performance um deles, porque é muito difícil combater com código nativo, uma vez que estamos essencialmente a correr a nossa aplicação num \textit{stripped down browser}, e o acesso às APIs nativas também é um problema, porque apesar de o \textit{Ionic} fornecer \textit{wrappers} a estas, nem todas são acessíveis através dum ambiente como este.

Esta \textit{framework} facilitou imenso o processo de desenvolvimento desta aplicação, que não era o foco da dissertação, mas permitiu mesmo assim apresentar algo onde se pudesse testar toda a aplicação num contexto mais semelhante ao mundo real, onde um utilizador pudesse realmente usufruir das abordagens aqui feitas. Neste capítulo não vamos abordar mais nem o \textit{wrapper} nem esta aplicação, deixando isso para o capítulo 8, onde vamos ver a aplicação em ação de modo a que as funcionalidades do \textit{middleware} fiquem mais claras.

\section{Detalhes}

Nesta secção irão ser detalhados alguns aspetos de uma importância acrescida, como o mecanismo de \textit{tunneling} que efetua a ligação entre vários \textit{hubs} e a \textit{api}, assim como elementos de meta-programação que permitem o crescimento do sistema em termos de dispositivos suportados, reduzindo o tempo de implementação e a manutenibilidade destes componentes. Também alguns aspetos de segurança, muito importante em contextos de IoT, vão ser explicados em mais detalhe.

\subsection{Hub}

Utilizando a framework web \textit{Spark}, conseguimos expor os nossos \textit{ThingServices} via uma API HTTP, como é possível ver no seguinte exemplo:

\begin{minted}{java}
ThingServiceFactory<Light> thingServiceFactory = new LightServiceFactory();
ThingController<Light> controller = new ThingController(
    thingServiceFactory,
    Light.class);

Spark.get("/devices/lights", controller::get);
Spark.put("/devices/lights", controller::put);
\end{minted}

Este exemplo faz uso da tal injeção de dependências explicada durante a fase da arquitetura, conseguindo assim ter apenas um controlador que faz a gestão de todos o tipos de dispositivos. Depois, são criadas duas rotas, uma utilizando o método GET e PUT, que se associam aos respetivos métodos no controlador. Aquela notação faz parte da API do Java 8, que permite passar métodos por referência.

Internamente, os métodos \textit{get} e \textit{put} recebem dois objetos da \textit{framework} utilizada, do tipo \textit{Request} e \textit{Response}, que têm todos os elementos do pedido HTTP proveniente do cliente assim como o objeto que representa a resposta a ser enviada. Com estes objetos é possível manipular cabeçalhos, \textit{query parameters}, entre outros.

No entanto, apesar dos mecanismos arquiteturais utilizados para simplificar o processo de adição de tipos e subtipos de dispositivos, ainda há um problema, sempre que é criado um novo tipo de dispositivo é preciso criar estas rotas manualmente, estando a aumentar a complexidade do código com \textit{boilerplate code}, código repetido sem grande significado. Portanto recorremos a elementos de meta-programação para resolver esta duplicação de código.

Meta-programação consiste então nas muitas maneiras que um programa consegue manipular o seu próprio funcionamento durante o tempo de execução. Em muitas linguagens orientadas a objetos, como o Java por exemplo, existe um mecanismo muito usado de meta-programação, chamado de reflexões, que permite a uma aplicação obter informações sobre as classes e elementos que a compõem, por exemplo, obter uma lista com o nome dos métodos que uma classe define, ou obter todas as classes que implementam um dado interface ou que estendem uma dada classe.

Para resolver este pequeno problema decidiu-se recorrer à biblioteca de \textit{reflections} do Java, que permite inferir durante o \textit{runtime}, a lista de classes que implementa o interface \textit{Thing}. Isto é bastante útil porque podemos utilizar os métodos definidos no interface, \textit{getFactory} e \textit{getStringRepresentation}, para automaticamente criar todas as rotas necessárias para o funcionamento do \textit{hub}.

Primeiro foi criada uma classe auxiliar com um método \textit{static} para obter todas as \textit{Things} do projeto.

\begin{minted}{java}
class ClassDiscoverer {
  public static Set<Class<? extends Thing>> getThings() {
    Reflections reflections = new Reflections("homewatch.things");
    return reflections.getSubTypesOf(Thing.class);
  }
}
\end{minted}

Este método obtêm todas as classes que implementam o interface \textit{Thing} e que estejam presentes no pacote no pacote \textit{homewatch.things}, esta pesquisa por pacote permite reduzir o tempo de execução da mesma.

Depois utilizamos este método para criar as rotas correspondentes aos 5 tipos de dispositivos que se implementaram.

\begin{minted}{java}
private static void deviceControllers() {
  ClassDiscoverer.getThings().forEach(klass -> {
    try {
      Thing t = klass.newInstance();
      ThingServiceFactory thingServiceFactory = t.getFactory();
      ThingController controller = new ThingController(thingServiceFactory, klass);

      Spark.get("/devices/" + t.getStringRepresentation(), controller::get);
      Spark.put("/devices/" + t.getStringRepresentation(), controller::put);
    } catch (InstantiationException | IllegalAccessException e) {
      LoggerUtils.logException(e);
    }
  });
}
\end{minted}

Este método é chamado nas classes de \textit{setup} do projeto, que tratam do processo de inicialização e configuração do pequeno servidor, e essencialmente percorre a lista de classes retornada pelo método \textit{getThings}, utilizando os dois métodos utilitários, \textit{getFactory} e \textit{getStringRepresentation} para criar as rotas de acesso aos serviços dinamicamente. Desta maneira, desde que os interfaces dos dispositivos retornem a \textit{factory} respetiva e afixando a descrição do dispositivo à rota de acesso.

Exemplificando o funcionamento deste mecanismo, se tivermos um dispositivo \textit{Light} onde o método \textit{getFactory} retorna um \textit{LightServiceFactory} e onde o método \textit{getStringRepresentation} retorna \textit{''lights''}, irá ser criado um \textit{controller} utilizando um \textit{LightServiceFactory}, e depois serão criadas duas rotas, \textit{''/devices/lights''} com os métodos GET e PUT.

Assim, conseguimos automaticamente instanciar todos os \textit{controllers} sem ter que repetir pedaços de código idênticos, que iriam acabar por se tornar muito numerosos com o crescimento de tipos de dispositivos.
\subsection{Middleware}

No que toca à API temos 3 detalhes importantes que vale a pena referir, os mecanismos de \textit{tunneling} para comunicar com o \textit{hub}, os elementos de metaprogramação utilizados nas \textit{triggered tasks}, e por fim, os mecanismos de \textit{background jobs} utilizados para lidar com as tarefas e outros aspetos.

\subsubsection{Tunneling}

Como já foi referido várias vezes durante esta dissertação, a comunicação entre o \textit{hub} e o \textit{middleware} iria ser feita através de um qualquer mecanismo de \textit{tunneling}, de modo a evitar tecnologias como encaminhamento de portas ou \textit{UPNP}, que têm processos de configuração complicados e algumas vulnerabilidades a nível de segurança.

Portanto, como tecnologia de \textit{tunneling} decidiu-se utilizar o \textit{ngrok} \footnote{\url{https://ngrok.com/}}, um software que permite expor serviços web locais, mesmo por de trás de \textit{firewalls} e NATs, disponibilizando este serviço num URL de acesso público. O \textit{ngrok} apesar de ser um software orientado a programadores, que desejam expor as suas aplicações web locais sem recorrer a \textit{port forwarding} para efeitos de demonstração, demonstrou ser uma das melhores soluções, nesta fase do desenvolvimento.

O \textit{ngrok} irá funcionar como um processo de background, que é executado sempre que o \textit{hub} é inicializado, expondo a porta 4567 num URL gerado automaticamente.

\begin{verbatim}
    ngrok http 4567

    https://370ef754.ngrok.io
\end{verbatim}

De notar que é gerado um URL cujo acesso pode ser feito utilizando o protocolo HTTPS, que assegura a encriptação dos dados que circulam neste túnel. Este URL deve ser guardado no recurso \textit{Home} presente na API do \textit{middleware}, no campo \textit{tunnel}, assim, o \textit{middleware} já consegue aceder à API exposta pelo \textit{hub}.

Neste preciso caso, o \textit{hub} possui antes um ficheiro de configuração com o túnel pré-definido

\begin{verbatim}
    web_addr: 0.0.0.0:4040
    region: eu
    tunnels:
      homewatch-hub:
        proto: http
        addr: 4567
\end{verbatim}

Este ficheiro define o endereço base da API do \textit{ngrok}, utilizada num mecanismo que vamos explorar já de seguida, a região do túnel, que pode ser na Europa ou nos Estados Unidos, e por fim, o túnel \textit{homewatch-hub}, expondo a porta 4567 que utiliza o protocolo HTTP.

A própria API do \textit{ngrok} é utilizada pelo \textit{hub} para fornecer o URL do túnel às aplicações clientes, para estas poderem criar o recurso \textit{Home} na API do \textit{middleware} fornecendo o túnel do seu \textit{hub}.

\begin{verbatim}
    http://homewatch-hub:4567/tunnel

    {
        "url" : "https://a10d2075.eu.ngrok.io"
    }
\end{verbatim}

Internamente o \textit{hub} efetua um pedido à API do \textit{ngrok} e extrai o URL do túnel, devolvendo-o no típico formato JSON. As aplicações podem sempre efetuar pedidos para aquele URL, uma vez que os \textit{routers} que os utilizadores possuem em casa têm na sua grande maioria um servidor DNS, que resolve IPs para os \textit{hostnames} das máquinas presentes na rede que adquiriram o seu IP via DHCP, e como o \textit{hostname} de todos os \textit{hubs} é \textit{homewatch-hub}, este processo torna-se perfeitamente possível. Este mecanismo é bastante útil para efetuar o \textit{setup} inicial do \textit{hub} com a API do \textit{middleware}, podendo obter o URL do \textit{hub} sem grandes problemas. No entanto, é preciso referir que obviamente só podemos recorrer a este mecanismo quando estamos na rede local do \textit{hub}, sendo impossível recorrer a esta operação numa rede externa.

Apesar da facilidade de uso e configuração dos túneis do \textit{ngrok}, temos problemas claros utilizando esta abordagem. De seguida iremos enumerar os principais problemas na utilização desta tecnologia.

\paragraph*{Geração Aleatória de URLs}

O primeiro contratempo encontrado logo antes de se montar o sistema à volta desta tecnologia foi a geração automática do URL, que obriga os utilizadores a reconfigurarem os dados da sua casa, na aplicação que acede ao \textit{middleware}, sempre que o \textit{hub} reinicia (regenerando assim o URL). A solução parcial para este problema foi o desenvolvimento do mecanismo de deteção à base dos servidores de DNS locais (dos \textit{routers} do utilizador) e da API do \textit{ngrok}, que se traduz no clique de um botão na aplicação para detetar e atualizar o túnel de uma casa. Apesar de ser uma boa solução, não é perfeita, havendo o inconveniente de as tarefas automatizadas não funcionarem enquanto que o URL do túnel não for atualizado, uma vez que qualquer tentativa de acesso ao \textit{hub} vai acabar com um erro.

\paragraph*{Limite de Conexões HTTP}

O outro problema consiste no limite de conexões HTTP imposto pelo \textit{ngrok} (20 por minuto), que põe em causa o funcionamento do \textit{polling} efetuado aos dispositivos através deste túnel no âmbito das tarefas automatizadas, uma vez que, cada \textit{triggered task} está sempre a monitorizar um dado dispositivo. A solução que resolveu este problema com sucesso foi a utilização de conexões persistentes HTTP \footnote{\textit{RFC HTTP Persistent Connections}: \url{https://www.w3.org/Protocols/rfc2616/rfc2616-sec8.html}}, que permite a reutilização de conexões HTTP, não causando a exaustão prematura das mesmas. Esta limitação também é evidente apenas no plano base (que é grátis), que foi utilizado para desenvolvimento.

\paragraph*{Segurança}

Por fim temos o problema de segurança, uma vez que apesar do tráfego dentro do túnel utilizar HTTPS (estando encriptado portanto), qualquer pessoa pode aceder ao túnel, caso tenha o URL de acesso, e enviar comandos para o \textit{hub}. De momento resolvemos esse problema manualmente, com o tal sistema de \textit{tokens} referido na arquitetura.

Num cenário ideal podia-se utilizar o \textit{ngrok link}\footnote{\textit{ngrok link}: \url{https://ngrok.com/product/ngrok-link}}, que oferece todos os benefícios deste software assim como outras funcionalidades que tornam a utilização desta tecnologia em massa muito mais acessível. Este plano resolve todos os problemas encontrados oferecendo ainda mecanismos avançados para a gestão em massa de túneis, que é exatamente o que acontece no nosso caso.
\begin{itemize}
    \item Gestão de túneis \textit{ngrok} em grande escala
    \item Reserva de endereços, basicamente garantindo a cada túnel um endereço único que nunca muda
    \item Encriptação avançada com \textit{tokens} e credenciais diferentes para cada túnel
    \item Utilização de HTTP/2 que oferece ganhos de performance consideráveis face ao HTTP
    \item IP \textit{whitelisting}, que permite garantir o acesso exclusivo apenas a certos IPs (neste caso apenas o IP da rede local do utilizador e o IP do servidor da API do \textit{middleware} eram \textit{whitelisted})
\end{itemize}

Todas estas vantagens eram oferecidas a partir de uma API poderosa exclusiva aos assinantes do \textit{ngrok link}. Esta tecnologia não foi utilizada uma vez que era orientada a negócios e a infraestruturas, não conseguindo ter acesso à mesma, no entanto, a escolha recaiu na versão base do \textit{ngrok} que permite demonstrar o funcionamento desta arquitetura e num cenário perfeito seria utilizado o \textit{ngrok link}.

No entanto, podemos concluir que, apesar de todas as desvantagens deste mecanismo, foi o único que se mostrou suficientemente fiável, tendo um custo de implementação bastante baixo. A utilização deste mecanismo simplificou bastante o processo de desenvolvimento e testes, permitindo alcançar a funcionalidade essencial do controlo remoto de dispositivos fora da rede local dos mesmos, mantendo um nível de segurança aceitável.

\paragraph*{Exemplo de utilização}

Utilizando este túnel, é possível então definir e obter os estados dos dispositivos na casa do utilizador, para isso basta chamar a API do \textit{hub} através deste URL gerado pelo \textit{ngrok}.

\begin{minted}{ruby}
  def get_status
    Curl.get(uri) do |http|
      http.headers["Content-Type"] = "application/json"
      http.headers["Authorization"] = home.token
    end
  end
  
  def send_status(status)
    Curl.put(uri, status.to_json) do |http|
      http.headers["Content-Type"] = "application/json"
      http.headers["Authorization"] = home.token
    end
  end
  
  private
  
  def connection_params
    connection_info.merge(subtype: subtype)
  end

  def uri
    home.tunnel + self.class.route + "?" + connection_params.to_query
  end
\end{minted}

Neste código acima apresentado definimos dois métodos auxiliares que ajudam a determinar a rota do dispositivo, onde fazemos uso do URL do túnel, combinado com a descrição da rota \textit{self.class.route}, seguido depois dos parâmetros de conexão, basicamente os parâmetros presentes no \textit{connection\_info} mais o subtipo, que são enviados na \textit{query string} do pedido. Depois fazemos uso do cliente HTTP \textit{curl} para então comunicar através do túnel, no caso do \textit{get\_status} basta apenas fazer um pedido GET para a rota gerada para obter o estado do dispositivo, já o método \textit{put\_status} basta fazer um pedido HTTP do tipo PUT com o novo estado desejado para o dispositivo. 

De resto, este túnel só voltar a ser utilizado na descoberta de dispositivos, onde basicamente o funcionamento é o mesmo.

\subsubsection{Meta-Programação}

Este componente, tendo sido desenvolvido em \textit{Ruby on Rails}, teve um desenvolvimento rápido e sem grandes problemas de repetição de código, devido às políticas utilizadas na \textit{framework}. Mesmo assim, houve uma pequena parte da aplicação onde se recorreu a meta-programação para efetivamente reduzir a quantidade de código desenvolvendo, melhorando a legibilidade e a manutenibilidade do mesmo.

O mecanismo está presente nas \textit{triggered tasks}, e é responsável por efetuar a comparação entre o estado fornecido na tarefa e o estado atual do dispositivo a monitorizar. Recorreu-se então a um mecanismo muito interessante presente no \textit{ruby}, que permite chamar um método num objeto dinamicamente, através de um método presente em todos os objetos do ruby, o \textit{send}.

\begin{minted}{ruby}
  class HelloWorld
    def hello(n)
      n.times { puts "Hello World" }
    end
  end

  hello_world = HelloWorld.new

  hello_world.send("hello", 5)
  
  #em condições normais faríamos assim:
  
  hello_world.hello(5)
\end{minted}

Este pequeno excerto de código demonstra a funcionalidade do método \textit{send}. Neste caso possuímos uma classe \textit{HelloWorld} com um método \textit{hello} que aceita um argumento \textit{n}, sendo que este método imprime o texto ''Hello World'' \textit{n} vezes quando invocado. Depois instanciámos um objeto \textit{HelloWorld}, invocando o método \textit{hello} com o argumento 5, recorrendo ao método \textit{send} em vez do método normal. Utilizando este método \textit{send} podemos chamar outros métodos, inclusive métodos privados, passando como argumento o nome do método e os restantes argumentos caso existam. De notar que podemos passar o nome do método que desejamos executar como \textit{string} ou símbolo, ou seja \textit{''hello''} ou \textit{:hello}.

No nosso caso recorremos a este mecanismo para executar uma operação de comparação variável, comparando um dado valor com outro com base num operador lógico à escolha. Isto porque o caso de uso das \textit{triggered tasks} era os utilizadores poder criar regras com base no estado de outros dispositivos com a maior liberdade possível, podendo criar tarefas como os seguintes exemplos:
\begin{itemize}
  \item Ligar a luz da garagem se o sensor de movimento colocado na mesma detetar movimento
  \item Desligar o sistema de ar condicionados interior se a temperatura registada pela estação meteorológica do jardim for inferior a 18ºC
\end{itemize}

Então, recorrendo a meta-programação, podemos pegar no \textit{comparator} presente em cada \textit{triggered task}, e utiliza-lo para comparar os estados do dispositivo monitorizado e do estado associado à tarefa.

\begin{minted}{ruby}
  def compare_remote_status(remote_status, status, comparator)
    status.each do |key, value|
      return false unless remote_status[key].send(comparator, value)
    end

    true
  rescue NoMethodError
    false
  end
\end{minted}

Este método compara ambos os estados, sendo que o \textit{remote\_status} é proveniente do dispositivo monitorizado e o \textit{status} é o estado alvo fornecido à tarefa em questão, utilizando um comparador que pode ser um qualquer operador lógico, como por exemplo, \textit{==} ou \textit{>=}, e quando a comparação tem um valor falso, terminamos de imediato o método, devolvendo um valor falso provando que ambos os estados não são iguais. Caso as comparações entre todos os elementos de ambos os estados tenham um valor verdadeiro, o valor retornado por este método também será verdadeiro, e, nesse caso, a tarefa será ativada.

Obviamente que este mecanismo é algo arriscado, uma vez que é propenso à injeção de instruções, porque o \textit{send} interpreta todo o código enviado como parâmetro, mas neste caso filtramos o valor do atributo \textit{comparator}, aceitando apenas os comparadores lógicos, eliminando este mesmo risco.

\subsubsection{\textit{Async Jobs}}

O último detalhe de relevo são as tarefas de \textit{background}, isto é, \textit{jobs} assíncronos, que são executados em segundo plano, para não sobrecarregar os servidores aplicacionais que tratam da lógica computacional e do tratamento de pedidos HTTP.

Por omissão, o \textit{Rails} já inclui uma biblioteca de tratamento de tarefas assíncronas, chamada de \textit{ActiveJob}, que permite delegar tarefas computacionais com um tempo de execução moderado ou longo para uma \textit{queue} onde serão processados assincronamente. Tarefas como o envio de emails, \textit{newsletters}, análise de dados, envio de notificações são normalmente executados utilizado esta biblioteca. 

Por omissão o \textit{ActiveJob} suporta dois modos de funcionamento, \textit{inline} ou \textit{async}, onde o primeiro executa as tarefas imediatamente (útil durante o desenvolvimento ou testes da aplicação), e o segundo executa as tarefas no mesmo processo só que numa \textit{thread pool} limitada, alcançando uma execução assíncrona. Ambos estes modos são úteis para desenvolvimento, mas muito limitados para executar durante a fase de produção, sendo necessário mecanismos mais elaborados para processar estas tarefas, sendo o \textit{DelayedJob} um deles. 

O \textit{DelayedJob} armazena os \textit{jobs} a executar numa base de dados, que pode ser a mesma base de dados utilizada pelo \textit{Rails}, tendo depois um ou vários processos consumidores, que consultam esta base de dados para irem executando tarefas assim que tiverem disponibilidade, funcionando como uma \textit{queue} essencialmente. Isto tem muitas vantagens a nível de performance, porque podemos separar completamente o processamento de tarefas assíncronas para outros processos e até outros sistemas, desde que tenham acesso à base de dados que contém as tarefas.

O \textit{DelayedJob} ainda tem uma outra funcionalidade interessante, oferecida por um plugin chamado de \textit{DelayedCronJob}, que permite criar tarefas recorrentes recorrendo a expressões \textit{cron}, basicamente oferecendo a funcionalidade das nossas \textit{Timed Tasks}, que também utilizam expressões \textit{cron}.

Vamos então observar um exemplo de um \textit{ActiveJob}:

\begin{minted}{ruby}
  class LongRunningJob < ApplicationJob
    queue_as :default
 
    def perform(*args)
      # tarefa complexa de longo tempo de execução
    end
  end
\end{minted}

Essencialmente, cada tarefa é uma classe, possuindo um método \textit{perform} que executa a tarefa em si com os dados argumentos. Estas tarefas podem ser executadas da seguinte maneira:

\begin{minted}{ruby}
  # executar assincronamente
  LongRunningJob.perform_later(args)
  
  # executar daqui a uma semana
  LongRunningJob.set(wait: 1.week).perform_later(guest)
\end{minted}

No entanto, o \textit{DelayedJob} oferece um mecanismo mais simplista, que é o que é usado no \textit{middleware}. Bastando prefixar qualquer chamada de um método com outro método chamado \textit{delay}:

\begin{minted}{ruby}
  thing.delay.put_status(on: false)
\end{minted}

Como podemos ver, isto torna a criação de \textit{async jobs} muito mais simplista. Com o plugin \textit{DelayedCronJob} podemos especificar a tal expressão \textit{cron}:

\begin{minted}{ruby}
  thing.delay(cron: "0 0 12 * *").put_status(on: false)
\end{minted}

Este mecanismo é utilizado então em 3 aspetos diferentes, gestão de tarefas \textit{triggered} e \textit{timed}, assim como os mecanismos de segurança do \textit{hub}, e, por fim, a aplicação de cenários, onde os dispositivos associados a um cenário são ativados assincronamente, para não bloquear os servidores aplicacionais.

\subsubsection{\textit{Service Objects}}

Como já foi referido na arquitetura, o \textit{middleware} irá recorrer a uma arquitetura MVC, bastante utilizada em paradigmas web, que também é a arquitetura base adaptada pela framework \textit{Ruby on Rails}. Nesta arquitetura dividimos o \textit{core} da aplicação em 3 tipos de componentes, \textit{models}, que representam a informação base do sistema, \textit{views}, que são a representação visual ou textual dos dados do sistema, e por fim, os \textit{controllers} que aceitam parâmetros externos fazendo alterações nos \textit{models}.

Utilizando o \textit{Rails} esta arquitetura torna-se muito simples de utilizar, tendo um \textit{model} para cada recurso da aplicação, como um \textit{User}, uma \textit{Home} ou uma \textit{Thing}, tendo cada um destes métodos para obtenção e manipulação de dados, recorrendo á biblioteca criada pela equipa do \textit{Rails}, o \textit{ActiveRecord}.

De seguida podemos ver um exemplo de um \textit{controller} muito simples:

\begin{minted}{ruby}
  class TodosController < ApplicationController
    # POST /todos
    def create
      todo = Todo.create!(todo_params)
        
       render json: todo
    end
    
    private
    
    def todo_params
      # whitelist params
      params.permit(:title, :created_by)
    end
  end
\end{minted}

Como podemos ver este código segue muitas das filosofias valorizadas pelo \textit{Rails}, sendo muito simples e fácil de compreender. Outra convenção é a nomeação dos métodos do \textit{controller}, devendo seguir a tal arquitetura REST, que já visitamos anteriormente no Capítulo 8.

No entanto, este \textit{controller} é muito básico, apenas tendo funcionalidades CRUD. Por vezes, certos métodos dos \textit{controllers} efetuam operações mais avançadas, por exemplo, no nosso caso sempre que um utilizador cria uma casa, a aplicação deve automaticamente efetuar a autenticação com o \textit{hub}. Se introduzisse-mos essa lógica no método \textit{create} do \textit{controller}, este ficaria muito grande e difícil de manter, portanto recorremos a um \textit{service object}.

\begin{minted}{ruby}
  def create
    create_home_service = CreateHome.new(user: current_user, params: home_params)
    home = create_home_service.perform

    if create_home_service.status
      render json: home, status: :created
    else
      render json: home.errors, status: :unprocessable_entity
    end
  end
\end{minted}

Como podemos constatar, mantemos uma legibilidade aceitável no método, extraindo a lógica mais complexa para um serviço. Os serviços são objetos comuns, que possuem um método \textit{perform}, responsável por executar a operação correspondente a este serviço. Caso a operação seja completada com sucesso, o serviço irá definir o seu atributo \textit{status} como verdadeiro.

Este serviço em particular, grava o registo da casa do utilizador na base de dados, efetua a autenticação no \textit{hub} gravando o \textit{token} no próprio registo da casa, e depois cria um \textit{DelayedJob} que regenera este \textit{token} todos os dias às 00:00.



