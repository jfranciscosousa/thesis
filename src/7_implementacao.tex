\chapter{Implementação}

Neste capítulo irá ser explicado todo o processo da implementação, desde as escolhas das tecnologias e às metodologias de desenvolvimento aos detalhes e pormenores de implementação de todos os componentes.

Após a definição da arquitetura, foram criados no sistema de controlo de versões remoto \textit{Github} para a implementação da plataforma:
\begin{itemize}
    \item Simuladores de dispositivos - \url{https://github.com/um-homewatch/api}
    \item \textit{Hub} - \url{https://github.com/um-homewatch/hub}
    \item \textit{Api} - \url{https://github.com/um-homewatch/stub-devices}
    \item \textit{Wrapper} da \textit{api} - \url{https://github.com/um-homewatch/js-wrapper}
    \item Aplicação cliente - \url{https://github.com/um-homewatch/app}
\end{itemize}

A ordem dos repositórios também ditam a ordem de implementação dos mesmos, seguindo a metodologia incremental e iterativa, onde os componentes foram implementados incrementalmente. O \textit{hub}, a \textit{api} e o \textit{wrapper} exibem todo o método de \textit{continuous integration}, recorrendo ao \textit{GitHub} e ao \textit{Travis}, enquanto que a \textit{api} é \textit{deployed} no \textit{Heroku}.

\section{Tecnologias}
De seguida serão apresentadas as tecnologias utilizadas para implementar a solução. Cada componente de software, o \textit{hub}, a \textit{api}, as simulações dos dispositivos e e aplicação cliente utilizam linguagens e \textit{frameworks} distintas. Aqui, iremos falar das motivações para a sua escolha.

\paragraph*{Hub}
Para o desenvolvimento deste componente recorreu-se à utilização da linguagem de programação Java, recorrendo à \textit{framework} Spark \footnote{\url{http://sparkjava.com}}.

Java foi uma escolha óbvia devido à experiência na linguagem e ao seu poder no que toca a genéricos, que permite atingir grandes níveis de reutilização de código, algo que será essencial neste caso, dado que vários dispositivos partilham o mesmo modo de comunicação por exemplo.

A \textit{framework} Spark foi escolhida devido à necessidade de expor um serviço HTTP sem ser necessário associar a um modelo arquitetural, por exemplo, algumas \textit{frameworks} são feitas para utilizar MVC. Neste caso, esta \textit{framework}, que até pode ser considerado uma biblioteca devido à sua simplicidade, tem apenas como função expor um servidor HTTP.

\paragraph*{Api}
Este componente, sendo o mais complexo de todos, necessita de uma \textit{framework} mais evoluída, portanto decidiu-se utilizar \textit{Ruby on Rails} e portanto a linguagem de programação \textit{Ruby}.

Esta \textit{framework} foi uma escolha óbvia, mais uma vez devido à experiência em trabalhos académicos passados, e também devido à simplicidade de utilização. Utilizando \textit{Ruby on Rails} podemos dedicar mais tempo a funcionalidades em vez de detalhes de implementação. Outra motivação é a existência de um bom suporte de \textit{background tasks}, que como já se falou anteriormente, era um desafio a enfrentar, para se poder implementar as tarefas automatizadas.

Utilizando esta \textit{framework}, tarefas como a camada de \textit{ORM}, filtragem de parâmetros dos pedidos HTTP, serialização de objetos para \textit{JSON} são todas tratadas pelo \textit{Rails}, podendo dar assim mais atenção a outros aspetos mais importantes da aplicação.

\paragraph*{Simulação de Dispositivos}

Para as simulações de dispositivos recorreu-se à linguagem Javascript com o \textit{runtime serverside} Node.JS, utilizando a framework \textit{express} \footnote{\url{https://expressjs.com/}}. A combinação de \textit{javascript} com o \textit{express} permite muito rapidamente desenvolver servidores web, uma vez que estes simuladores irão fornecer APIs HTTP.

Também se utilizou a biblioteca \textit{coap-router} \footnote{\url{https://github.com/MagicCube/coap-router}}, para criar os dispositivos que utilizam CoAP em vez de HTTP, como protocolo de comunicação.

\section{Detalhes}

Nesta secção irão ser detalhados alguns aspetos de uma importância acrescida, como o mecanismo de \textit{tunneling} que efetua a ligação entre vários \textit{hubs} e a \textit{api}, assim como elementos de meta-programação que permitem o crescimento do sistema em termos de dispositivos suportados, reduzindo o tempo de implementação e a manutenibilidade destes componentes.

\subsubsection{Tunneling}

falar do ngrok e como é utilizado e as suas vantagens

\subsection{Meta-Programação}

falar dos elementos de meta programação no hub (reflections e autoloading das Thing classes)

falar também das comparações e definições em runtime nas triggered tasks

\subsubsection{Hub}

\subsubsection{Api}