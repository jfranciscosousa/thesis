\chapter{Contextualização}

Nesta última década vem se observando um crescimento bastante acentuado do mercado de dispositivos orientados ao espaço casa. Dispositivos que permitem automatizar diversas áreas de uma casa, como os estores, as portas, as luzes, o aquecimento e até sistemas de vídeo-vigilância. O número de dispositivos e as funcionalidades dos mesmos são bastante diversos, e todos têm a capacidade de serem controlados através de sistemas de informação, como por exemplo, \textit{smartphones}. 

Tais dispositivos compõem aquilo a que muitos chamam de ''casas inteligentes'', que é uma das área da ''Internet of Things'', ou IoT abreviadamente, que basicamente representa a rede de dispositivos físicos como sensores, eletrodomésticos, atuadores entre outros. Juntamente às ''casas inteligentes'' juntam-se outras áreas no top 5 da IoT, as plataformas cloud de IoT, automação industrial, industria energética e ''connected cities''. 

De acordo com um questionário feito pelo IEEE \cite{ieeesurvey} a vários profissionais das tecnologias de informação, cerca de 46\% dos correspondentes estavam a trabalhar na área das IoT e outros 29\% tinham planos para desenvolver soluções IoT nos próximos 18 meses. Isto vem confirmar a crescente tendência em optar estas novas tecnologias, que pretendem melhorar a interação entre os vários dispositivos ''inteligentes'' que compõem o mundo.

Estes novos dispositivos inteligentes são normalmente acompanhados por aplicações clientes feitas pelos próprios fabricantes, que permitem, no mínimo, ativar e desativar o dispositivo e recolher a informação sensorial do mesmo, caso tenha esta funcionalidade. Nem todos os dispositivos recolhem informação do meio, por exemplo, uma fechadura eletrónica ou um dispositivo de iluminação. 

No entanto, com a crescente utilização de dispositivos nas residências dos utilizadores, a utilização de várias aplicações proprietárias para gerir cada um destes dispositivos torna-se inconveniente \cite{iot-survey-issues}. Daí a conceção de frameworks para gestão de dispositivos orientados à \textit{home-automation}. Um dos exemplos mais conhecidos é o Apple Home, que permite controlar os dispositivos instalados numa residência através de qualquer dispositivo iOS. O utilizador adiciona os dispositivos ao Homekit, a base de dados por detrás do Apple Home, e passa a ter um local centralizado onde pode gerir os aparelhos da sua residência. No entanto, esta framework não está disponível noutros sistemas operativos, e apenas suporta uma distinta gama de dispositivos, não sendo totalmente aberta. 

Esta variedade de dispositivos e tecnologias nesta área provoca também alguns entraves ao desenvolvimento de novas soluções, uma vez que, não existem soluções “standard” para o seu desenvolvimento. Existem implementações que utilizam vários tipos de tecnologia, tornando difícil a interoperabilidade entre estes novos produtos. De notar que uma das filosofias da IoT é a interconectividade entre estes dispositivos e sem interoperabilidade entre dispositivos, torna-se impossível de alcançar tal coisa.

Algumas tecnologias têm vindo a surgir recentemente, como o Raspberry PI, que é uma excelente placa de prototipagem, que possui um circuito com 24 pinos de GPIO, que permite controlar qualquer tipo de aparelhos através desta placa. No campo de bibliotecas de sofware, o Eclipse SmartHome parece ser uma tecnologia interessante, semelhante ao Homekit, mas orientada para programadores e para o desenvolvimento de novos produtos, e o Google Brillo e o Weave parecem tecnologias interessantes orientadas à IoT baseadas no ecossistema da Google.

Este exemplo da Google demonstra um cenário ''cloud-based'', o que é uma grande vantagem, uma vez que parte do esforço computacional é retirado do utilizador e dos seus equipamentos, e também centraliza a sua informação num serviço que pode ser acedido de qualquer lugar, desde que possua ligação à Internet. A quantidade de oferta de serviços de alojamento cloud torna o desenvolvimento nesta área bastante mais conveniente. Além disso, provedores de serviço cloud como a Amazon oferecem serviços especializados para IoT, assim como integração total com o resto dos seus serviços, mais uma vez demonstrando como o futuro da IoT reside na cloud.

De referir, que apesar disto ser um projeto a visar o espaço casa, as conclusões e resultados podem ser aplicados também ao espaço empresarial e industrial, onde os casos de uso são semelhantes, mas com um maior volume de dados e por vezes interações especificas, como as se verificam com os equipamentos industriais. Um caso de uso onde a \textit{home automation} encaixa bastante bem é no espaço empresarial, por exemplo, nos escritórios, onde a regulação de equipamentos como ar condicionados, alarmes, sistemas de videovigilância, portas e fechaduras são aspetos de grande importância.