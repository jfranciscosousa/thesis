\chapter*{Resumo}

Com a crescente utilização e adoção de dispositivos orientados à \textit{Internet of Things}, especialmente dispositivos orientados ao espaço casa, crescem também o número de protocolos, APIs e \textit{frameworks} desenvolvidos pelos fabricantes para complementar os seus dispositivos. Este elevado número de soluções, sem a adoção de um standard geral que facilita o desenvolvimento de aplicações para esta área, serve de motivação para esta dissertação.

Com isto em mente, esta dissertação tem como objetivo o desenvolvimento de uma camada de \textit{middleware}, que faz a ligação entre os mais diversos dispositivos e as aplicações clientes, abstraindo os dispositivos e as funcionalidades, facilitando imenso o desenvolvimento destas aplicações. Este \textit{middleware} também deverá oferecer algumas comodidades a nível de utilização, sob a forma de mecanismos de automação e definição de cenários.

A dissertação irá expor o estado da tecnologia nesta área, identificando as mais valias e também os possíveis problemas das mesmas. Também será revisto algum trabalho de investigação na área, de modo a entender os esforços já feitos. Feito o estudo sobre o estado de arte, iremos proceder ao desenho e conceção da arquitetura do \textit{middleware}, passando para o desenvolvimento do mesmo. Por fim, será feito um caso de estudo, sobre a forma de uma aplicação mobile, que faz uso do \textit{middleware}, demonstrando todas as suas funcionalidades e capacidades de interoperabilidade.